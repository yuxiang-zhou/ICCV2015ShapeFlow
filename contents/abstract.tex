Uses Abstract from Nondas' paper "Automatic Construction of Deformable Models In-The-Wild" as place holder with some modification. 
Deformable objects are everywhere. Faces, ears, bodies,
bottles etc. Recently, there has been a wealth of research
on training deformable models for object detection,
part localization and recognition using annotated data. In
order to train deformable models with good generalization
ability, a large amount of carefully annotated data is required,
which is a highly time consuming and costly task.
We propose the first - to the best of our knowledge - method
for construction of dense deformable models using images
captured in totally unconstrained conditions, recently
referred to as “in-the-wild”. The only requirements of the
method are drawings of image features. The object detector can be as simple as the
Viola-Jones algorithm (e.g. even the cheapest digital camera
features a robust face detector). The 2D shape model
can be created by using only a few shape examples with deformations.
In our experiments on facial deformable models,
we show that the proposed automatically built model
not only performs well, but also outperforms discriminative
models trained on carefully annotated data. To the best of
our knowledge, this is the first time it is shown that an automatically
constructed model can perform as well as methods
trained directly on annotated data.